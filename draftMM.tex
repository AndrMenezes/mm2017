	\documentclass[12pt,a4paper,final]{article}
%%%%%%%%%%%%%%%%%%%%%%%%%%%%%%%%%%%%%%%%%%%%%%%%%%%%%%%%%%%%%%%%%%%%%%%%%%%%%%%%%%%%%%%%%%%%%%%%%%%%%%%%%%%%%%%%%%%%%%%%%%%%%%%%%%%%%%
\usepackage{verbatim,epsfig,theorem,fancybox,amsmath,amsfonts,amsbsy,ae,indentfirst,float,pdfpages,multicol}
\usepackage{geometry,array,enumerate,setspace,bm,multirow,arydshln, booktabs, color,scalefnt, listings}
\usepackage[hidelinks, brazil]{hyperref}
\usepackage[utf8]{inputenc}
\usepackage[brazil]{babel}
%\usepackage[alf]{abntex2cite}	% Citações padrão ABNT
\usepackage[alf,abnt-etal-list=0,abnt-etal-cite=3]{abntex2cite}
\urlstyle{tt}
\hyphenation{li-te-ra-tu-ra}
\geometry{a4paper,nohead,left=2.5cm,right=2.5cm,bottom=2.5cm,top=2.5cm}

\DeclareRobustCommand{\rchi}{{\mathpalette\irchi\relax}}
\newcommand{\irchi}[2]{\raisebox{\depth}{$#1\chi$}} % inner command, used by \rchi

\setcounter{secnumdepth}{4}
\lstset{breaklines=true}
\lstset{basicstyle=\scriptsize\tt}
\begin{document}
\begin{titlepage}
	\begin{flushleft}
		\Large\textbf{Universidade Estadual de Maringá} \\
		\large\textbf{Departamento de Estatística} \\
		\textbf{Disciplina:} 8076 -- Modelos Mistos\\
		\textbf{Professora:} Dra. Terezinha Aparecida Guedes \\
		\textbf{Acadêmico:} André Felipe Berdusco Menezes 
	\end{flushleft}
	
	\vspace{180pt}
	\begin{center}
		\LARGE\textbf{Modelos Mistos: aplicação em dados de frutos de laranja doce}
	\end{center}
	
	\vspace{78pt}
	\begin{center}
		\vspace{180pt}
		\textrm{Maringá \\ 
			Agosto de 2017}
	\end{center}
\end{titlepage}
	
\onehalfspacing
\tableofcontents

\newpage

\section{Introdução}
Em diversas áreas, é comum na prática o uso de experimentos longitudinais. 
Segundo \citeonline{Verbeke1997} um experimento onde cada sujeito é observado 
sob várias condições experimentais, é denominado de medidas repetidas. Por outro lado,
os mesmos autores definem dados longitudinais quando o experimento mede repetidamente 
ao longo do tempo cada individuo.

Neste trabalho iremos analisar os dados provenientes de um estudo longitudinal,
onde se avaliou em cinco tempos diferentes o diâmetro do pedúnculo da laranja doce sob
o efeito de diferentes dose. As avaliações foram realizadas semanalmente
aos 0 (avaliação inicial), 7, 14, 21 e 28 dias. Ressalta-se que nos materiais disponibilizados
não há menção sobre o significado das doses, isto é, o que foi aplicado. Conforme o pesquisador, 
responsável pelos dados, nas doses utilizadas, não se esperava diferença significativa, 
porque as doses foram de regulador hormonal, somente para influenciar a cicatrização e interferir na doença.

Uma vez que a estrutura original dos dados não se adequavam com a metodologia estudada na disciplina
Modelos Mistos (8076), tal estrutura foi modificada. Nesse sentido, além da variável resposta (diâmetro), 
as covariáveis avaliação (tempo em dias) e tratamento (doses) foram medidas em cada árvore (fruto) ao longo do tempo.
Logo, o objetivo deste estudo se reduz em comparar o comportamento do diâmetro de cada fruto ao longo do tempo (avaliações)
e identificar se existe influência das doses, tempo (avaliações) bem como da interação entre estes fatores. 

Além desta introdução o presente trabalho estado organizado da seguinte forma. 
Na Seção 2 discuto a teoria dos modelos mistos aplicada sob a perspectiva da análise 
de dados longitudinais, além disso também descrevemos as possíveis estruturas dos modelos 
para responder os objetivos do pesquisador. Uma análise descritiva do conjunto de dados, resultados 
da modelagem e discussões são expostas na Seção 3. Por fim, algumas considerações gerais sobre o trabalho 
finalizam este relatório na Seção 4.

\newpage
\section{Metodologia}
Esta seção é dedicada a um breve apanhado sobre a teoria dos modelos mistos. Maiores detalhes sobre os modelos
mistos podem ser encontrados em: 
\citeonline{Pinheiro1994}
\citeonline{Verbeke1997},
\citeonline{Pinheiro2000} 
\citeonline{Verbeke2000}, 
\citeonline{Diggle2002}, 
\citeonline{Littell2006}
\citeonline{Bates2010},
\citeonline{Demidenko2013} e 
\citeonline{Singer2017}.


Os modelos lineares mistos são definidos, em uma forma compacta, da seguinte forma:

\begin{equation}\label{eq:definicao}
\mathbf{y} = \mathbf{X}\,\bm{\beta} + \mathbf{Z}\,\mathbf{u} + \mathbf{e}
\end{equation}
em que:
\begin{itemize}
	\item $\mathbf{y}$ é um vetor $ n \times 1 $ das variáveis;
	\item $\mathbf{X}$ é uma matriz $ n \times p $ de especificação dos efeitos fixos;
	\item $\bm{\beta}$ é um vetor $ p \times 1 $ de parâmetros dos efeitos fixos;
	\item $\mathbf{Z} = \left[\mathbf{Z}_1, \ldots, \mathbf{Z}_b\right]$, onde $ \mathbf{Z}_i $ é uma matriz
	$ n \times q_i $ de especificação para o $i$-ésimo efeito aleatório;
	\item $\mathbf{u}$ é um vetor $ q \times 1$ de efeitos aleatórios, onde $ \mathbf{u}_i $ é outro vetor
	$ q_i \times 1 $ tal que $q = \sum\limits_{i=1}^{b}\,q_i$, com $\mathbb{E}(\mathbf{u}) = \bm{0}$;
	\item $\mathbf{e}$ é um vetor $ n \times 1$ de erros aleatórios, com $\mathbb{E}(\mathbf{e}) = \bm{0}$.
\end{itemize}

Sob esta formulação assumimos que $\mathbf{u}$ e $\mathbf{e}$ são independentes e seguem distribuição Normal multivariada
de modo que:
\begin{equation}\label{eq:dist}
\left[\begin{array}{c}
\mathbf{u} \\
\mathbf{e}
\end{array} \right] \sim 
\mathcal{N}\left(
\left[\begin{array}{c}
\bm{0} \\
\bm{0}
\end{array}
\right],
\sigma^2
\left[\begin{array}{cc}
\mathbf{G}(\bm{\gamma}) & \bm{0} \\
\bm{0} & \mathbf{R}(\bm{\rho})                  
\end{array}
\right]
\right)
\end{equation}
em que $ \bm{\gamma} $ e $ \bm{\rho} $ são vetores $ r \times 1 $ e $ s \times 1 $ (com $s \leq n\,(n+1) / 2$)
desconhecidos dos parâmetros de variâncias correspondentes a $ \mathbf{u} $ e $ \mathbf{e} $, respectivamente.
Conforme \citeonline{Verbeke2000} escrevemos a matriz de variância-covariância dos dados, $ \mathbf{y} $, como:

\begin{equation}\label{eq:var}
\textrm{Var}(\mathbf{y}) = \sigma^2\,\left(\mathbf{Z}\,\mathbf{G}\,\mathbf{Z}^\top + \mathbf{R}\right) 
= \sigma^2\,\mathbf{H}
\end{equation}

A matriz $ \mathbf{H} $ consiste de duas componentes que são utilizadas para modelar a heterocedasticidade e 
correlação, sendo elas: um componente de efeito aleatório $ \mathbf{Z}\,\mathbf{G}\,\mathbf{Z} $ e um componente
dentro do grupo $ \mathbf{R} $. Ressalta-se aqui, que a componente intra grupo, $ \mathbf{R} $, em algumas 
aplicações, é utilizada diretamente para modelar a matriz de variância-covariância, sem a necessidade de incorporar
efeitos aleatórios no modelo para explicar a dependência entre as observações.

Como discutido em \citeonline{Singer2017} grande parte do esforço empregado na modelagem de dados com medidas repetidas
se concentra na estrutura de covariância. No contexto dos modelos mistos, a covariância entre as observações
obtidas em uma mesma unidade amostral poderá ser modelada diretamente por meio da matriz $ \mathbf{R} $.
Além disso, \citeonline{Diggle2002} ressalta que os modelos para a matriz $ \mathbf{R} $ podem incluir 
três possíveis fontes de variação na estrutura de covariância: 
aquelas correspondente aos efeitos aleatórios, à correlação serial e aos erros de medida. 
Aqui deve ser ressaltado que para os dados em estudo iremos considerar as estruturas \emph{Compound Symmetry},
Exponencial e Gaussiana. Para informações sobre outras estruturas o leitor pode consultar as referencias citadas
anteriormente ou ainda o manual da \texttt{PROC MIXED} do \textsf{SAS} \cite{MIXED}.

Dado o modelo formulado \eqref{eq:definicao} é necessário estimar seus parâmetros. Neste caso, temos 
parâmetro fixos, $ \bm{\beta} $, e os parâmetros de variâncias dos efeitos aleatórios $\mathbf{u}$
e do erro aleatório $\mathbf{e}$. Diferente métodos de estimação foram propostos, mas seguramente os mais
utilizados são os métodos da máxima verossimilhança (ML) e máxima verossimilhança restrita (RML).

Para obter as estimativas das matrizes $ \mathbf{G} $ e $ \mathbf{R} $ maximizamos as funções associada a ML e RML. 
As correspondentes funções de log-verossimilhanças são definidas, respectivamente, por:
\begin{equation}\label{eq:loglik}
\begin{split}
\text{ML:}&\qquad 
\ell(\mathbf{G}, \mathbf{R}) = -\dfrac{1}{2}\,\log \left|\mathbf{V}\right| - 
\dfrac{1}{2}\,\mathbf{r}^\top\,\mathbf{V}^{-1}\,\mathbf{r} - \dfrac{n}{2}\,\log\left(2\,\pi\right)\\
\text{RML:}&\qquad 
\ell_R(\mathbf{G}, \mathbf{R}) = -\dfrac{1}{2}\,\log \left|\mathbf{V}\right| - 
\dfrac{1}{2}\,\log \left|\mathbf{X}^\top\,\mathbf{V}^{-1}\,\mathbf{X}\right| - 
\dfrac{1}{2}\,\mathbf{r}^\top\,\mathbf{V}^{-1}\,\mathbf{r} - \dfrac{n - p}{2}\,\log\left(2\,\pi\right)\nonumber
\end{split}
\end{equation}
em que $ \mathbf{r} = \mathbf{y} - \mathbf{X}\,\left(\mathbf{X}^\top\,\mathbf{V}^{-1}\,\mathbf{X}\right)^{-1}
\,\mathbf{X}^\top\,\mathbf{V}^{-1}\,\mathbf{y}$ e $ p $ é o posto de $ \mathbf{X} $.

Para obter as estimativas dos efeitos fixos $ \bm{\beta} $ e as predições dos efeitos aleatórios 
$ \bm{\gamma} $ o método padrão é resolver as \emph{equações de modelos mistos} definidas por \citeonline{Henderson1984}:
\begin{equation}\label{eq:emm}
\left[\begin{array}{cc}
\mathbf{X}^\top\,\mathbf{\widehat{R}}^{-1}\,\mathbf{X} & \mathbf{X}^\top\,\mathbf{\widehat{R}}^{-1}\,\mathbf{Z} \\
\mathbf{Z}^\top\,\mathbf{\widehat{R}}^{-1}\,\mathbf{X} & \mathbf{Z}^\top\,\mathbf{\widehat{R}}^{-1}\,\mathbf{Z} + 
\mathbf{\widehat{G}}^{-1}
\end{array}\right]\,
\left[\begin{array}{c}
\widehat{\bm{\beta}} \\
\widehat{\bm{\gamma}} \\
\end{array} \right] =
\left[\begin{array}{cc}
\mathbf{X}^\top\,\mathbf{\widehat{R}}^{-1}\,\mathbf{y}\\
\mathbf{Z}^\top\,\mathbf{\widehat{R}}^{-1}\,\mathbf{y}
\end{array}
\right]
\end{equation}
As soluções podem ser escritas como:
\begin{equation}\label{eq:emm2}
\begin{split}
\widehat{\bm{\beta}}  &= \left(\mathbf{X}^\top\,\mathbf{\widehat{V}}^{-1}\,\mathbf{X}\right)^{-1}\,\mathbf{X}^\top\,
\widehat{\mathbf{V}}^{-1}\,\mathbf{y}\\
\widehat{\bm{\gamma}} &= \widehat{\mathbf{G}}\,\mathbf{Z}^\top\,\widehat{\mathbf{V}}^{-1}\,
\left(\mathbf{y} - \mathbf{X}\,\bm{\beta}\right)
\end{split}
\end{equation}

\newpage
Inferências sob os parâmetros do modelo \eqref{eq:definicao} são realizadas sob o paradigma da teoria da verossimilhança,
em particular utilizando o fato de que os estimadores de máxima verossimilhança são assintoticamente normais, com 
matriz de variância-covariância dada pela inversa da informação de Fisher. Para comparação de modelos encaixados
procedeu-se com o teste da razão de verossimilhança, a qual sua estatística é definida por:
\begin{equation}
\label{eq:trv}
S_{LR} =  2\left(\ell_c - \ell_r\right)
\end{equation}
em que $ \ell_r $ e $ \ell_c $ são os valores das funções de log-verossimilhanças avaliadas sob hipótese nula e alternativa, respectivamente. A estatística $ S_{LR} $ tem distribuição assintótica qui-quadrado com $ p $ graus de liberdade, sendo $ p $ o número de restrições impostas.

Por outro lado, para comparar as estruturas de covariância os critérios de informação de Akaike (AIC) 
e Bayesiano (BIC) foram utilizados. Eles são definidos, respectivamente, por:
\begin{equation}
\textrm{AIC} = -2\log (L)+\dfrac {2np}{n-p-1} \qquad  \text{ e } \qquad
\textrm{BIC} = -2\log (L)+p\log (n) \qquad 
\end{equation}
em que $L$ é o valor da função de verossimilhança avaliada nas estimativas de máxima verossimilhança,
$n$ é o número de observações e $p$ é o número de parâmetros estimados.

Por fim, ressalta-se que toda a análise foi conduzida no software \textsf{R} \cite{RCT2016} com auxilio das
bibliotecas \texttt{lme4} \cite{Bates2015} e \texttt{nlme} \cite{Pinheiro2016}
para ajustar os modelos, comparação dos mesmos e análise de resíduos.

\newpage
\section{Resultados e Discussões}
Como apontado por \citeonline{Bates2010} uma representação gráfica interessante para avaliar 
em dados longitudinais é verificar se existe alguma tendência da variável resposta
ao longo do tempo conforme o sujeito. No entanto para os dados considerados neste 
trabalho este gráfico não forneceu informações relevantes, uma vez que 80 sujeitos
foram observados ao longo das cinco avaliações. Nesse sentido, a análise descritiva
ficou restrita a verificar o comportamento da variável resposta de acordo com os níveis 
do tratamento e ao longo do tempo. Ou seja, sem considerar a trajetória do individuo ao 
longo do tempo.

Na \autoref{fig:bp1} é apresentada o comportamento da variável resposta conforme as doses
observadas. Os boxplots indicam que a variabilidade entre os tratamento, bem como a mediana
são muito similares. 
\begin{figure}[H]
\centering
\includegraphics[scale = 0.6]{boxplot-trat.pdf}
\caption{Boxplots do diâmetro conforme as doses aplicadas.}
\label{fig:bp1}
\end{figure}

Por outro lado, o comportamento do diâmetro ao longo do tempo é representado na \autoref{fig:bp2}.
Inicialmente, nota-se diferenças entre os boxplots, tanto em posição (mediana) quanto em dispersão (amplitudes).
Curiosamente, não existe uma tendência de crescimento ou decrescimento ao longo do tempo da variável resposta
diâmetro do fruto. Os dados indicam que a maior média do diâmetro ocorre após 21 semana, por outro lado,
as menores médias são na semana 7 e 28. Embora exista diferenças entre os tempos, este fato comprova a falta
de tendência ao longo do tempo.
 
\begin{figure}[H]
\centering
\includegraphics[scale = 0.6]{boxplot-tempo.pdf}
\caption{Boxplots do diâmetro conforme as avaliações (tempo).}
\label{fig:bp2}
\end{figure}

Finalizando a análise descritiva, apresentamos na \autoref{fig:bp3} o gráfico da interação entre as avaliações 
(tempo) e as doses aplicadas. Pode-se notar que não existe fortes diferenças entre os boxplots dentro de cada tempo,
por exemplo, fica evidente que 7 dias após a avaliação o comportamento dos boxplots são bastante similares. 
Outra vez, pode-se observar pela linha tracejada e os pontos, a falta de tendência da variável resposta ao longo do
tempo, bem como a diferença da  mesma a medida que as avaliações ocorrem.

\begin{figure}[H]
	\centering
	\includegraphics[scale = 0.6]{boxplot-tempo-trat.pdf}
	\caption{Boxplots do diâmetro conforme as avaliações (tempo) e as doses aplicadas.}
	\label{fig:bp3}
\end{figure}

Considerando a descrição dos dados exposta e a teoria dos modelos mistos discutido nas seções anteriores foi considerado
um modelo misto afim de explicar a variabilidade dos dados e responder as perguntas do pesquisador. Algumas questões
de interesse, entre outras, são: existe diferença entre as doses? O diâmetro do fruto sofre influencia do tempo? 
Existe alguma combinação entre tempo e tratamento que fornece maior(menor) resposta em média. Deve ser mencionado,
que algumas dessas indagações podem ser vistas de forma empírica pela análise descritiva aqui apresentada. No entanto,
um modelo estatístico que considerar erros aleatórios deve ser empregado afim de formalizar os resultados expostos
na análise exploratória.

Seja $ Y_i $ o valor do diâmetro da $ i $-ésima árvore o modelo considerado inicialmente pode ser escrito por:
\begin{equation}\label{eq:mod1}
Y_i = \mu + \alpha(\texttt{dose}_i) + \beta(\texttt{tempo}_i) + \gamma(\texttt{tempo}_i\,\texttt{dose}_i) + 
\delta(\texttt{arvore}_i) + \varepsilon_i
\end{equation}
em que $ i = 1, \ldots, 400 $, $\delta(\texttt{arvore}_i) \sim \mathcal{N}(0, \sigma^2_\delta)$ e $\varepsilon_i \sim \mathcal{N}(0, \sigma^2)$ e todos $\delta$'s e $\varepsilon$'s são independentes. Ressalta-se que o tempo (avaliação) foi
considerada como covariável e não fator.

A estrutura de covariância deste modelo pode ser expressa por:
$$
\textrm{Cov}(y_{i}, y_{j}) =
\begin{cases}
0                     & \text{, se } \texttt{arvore}_{i} \neq \texttt{arvore}_j \text{ e } i \neq j\\
\sigma^2_\delta            & \text{, se } \texttt{arvore}_{i} = \texttt{arvore}_j \text{ e } i \neq j\\
\sigma^2_\delta + \sigma^2 & \text{, se } i = j
\end{cases}
$$
Esta estrutura de variância, sugere que duas observações de arvores
diferentes não estão correlacionadas, e duas observações da mesma
arvores estão correlacionadas positivamente com o coeficiente de
correlação dado por:
\begin{equation}\label{eq:rho}
\rho = \dfrac{\sigma^2_\delta}{\sigma^2_\delta + \sigma^2}
\end{equation}

Os resultados da ANOVA para os efeitos fixos do modelo \eqref{eq:mod1} são apresentados na \autoref{tab:anova1}.
Pelos resultados da \autoref{tab:anova1} observa-se que o efeito da dose e a interação entre dose e tempo (avaliação)
não foi significativo. Esse resultado já era esperado, pois como apresentado nas análises descritivas, as doses 
assim como sua interação com o tempo não proporcionou grupos com fortes diferenças (ver \autoref{fig:bp1} e 
\autoref{fig:bp3}).

\begin{table}[H]
	\centering
	\caption{Resultados da ANOVA para os efeitos fixos.}
	\onehalfspacing
	\label{tab:anova1}
	\begin{tabular}{lrrrr}\toprule
		Parâmetro& G.L. num  & G.L. den & F & valor-p \\ \midrule
Intercepto &     1 & 316 & 119708.2262 & 0.0000 \\ 
Dose &     3 & 76 & 1.2644 & 0.2927 \\ 
Tempo &     1 & 316 & 5.4009 & 0.0208 \\ 
Dose:Tempo &     3 & 316 & 0.2474 & 0.8631 \\ \bottomrule
	\end{tabular}
\end{table}

Como a interação não foi significativa, ela será retirada do modelo \eqref{eq:mod1}. Os resultados da ANOVA sem a 
interação estão expostos na \autoref{tab:anova2}. Pode-se notar que mesmo sem a interação o efeito da dose sobre
o diâmetro do fruto de laranja não é significativo, em outras palavras dizemos que o comportamento do diâmetro da laranja,
neste estudo, é independente da dose aplicada.

\begin{table}[H]
\centering
\caption{Resultados da ANOVA para os efeitos fixos.}
\onehalfspacing
\label{tab:anova2}
\begin{tabular}{lrrrr}\toprule
	Parâmetro& G.L. num  & G.L. den & F & valor-p \\ \midrule
		Intercepto &     1 & 319  & 119708.2355 & 0.0000 \\ 
		Dose &     3 & 76  & 1.2644 & 0.2927 \\ 
		Tempo &     1 & 319  & 5.4394 & 0.0203 \\ \bottomrule
\end{tabular}
\end{table}

Tendo em vista os resultados discutidos e apresentados nas Tabelas \ref{tab:anova1} e \ref{tab:anova2} o modelo proposto
se reduz a um modelo de regressão linear simples com efeito aleatório do sujeito (árvore). Sua especificação é dada por:
\begin{equation}\label{eq:mod2}
Y_i = \beta_0 + \beta_1\,\texttt{tempo}_i + \delta(\texttt{arvore}_i) + \varepsilon_i
\end{equation}
em que $\beta_0$ é o intercepto, $ \beta_1 $ o coeficiente angular devido ao tempo (avaliações), $\delta(\texttt{arvore})$
o efeito aleatório devido ao sujeito e $\varepsilon_i$ são os erros aleatórios. Além disso, assumimos que
$\delta(\texttt{arvore}_i) \sim \mathcal{N}(0, \sigma^2_\delta)$ e $\varepsilon_i \sim \mathcal{N}(0, \sigma^2)$ e todos $\delta$'s e $\varepsilon$'s são independentes. 

A estrutura de covariância do modelo \eqref{eq:mod2} é a mesma discutida anteriormente, ela é denominada 
na literatura como \emph{Compound Symmetry} \cite{Verbeke1997}. Nesta estrutura 
temos que duas medidas de um mesmo individuo estão correlacionadas não importa a distância 
(em tempo ou espaço) que elas foram medidas. Para os dados em estudo, uma estrutura intuitiva para a 
covariância dos erros é aquele onde
a correlação entre duas medidas de um mesmo individuo (arvore) dependem de quão distante elas foram tomadas. 
Neste sentido, foi outras estruturas, denominadas de Exponencial e Gaussiana, também foram consideras. 
Maiores detalhes sobre estas estruturas de correlação podem ser encontra em \citeonline{Diggle2002} e 
\citeonline{Verbeke1997}.

Para comparar as estruturas \emph{Compound Symmetry} (CS), Exponencial (Exp) e Gaussiana (Gaus) os critérios de informação
AIC e BIC foram utilizados. Uma vez que modelos com as estruturas Exp e Gaus são modelos aninhados com a estrutura CS,
o teste da razão de verossimilhança foi utilizado nestes casos. 

Os critérios utilizados para comparar as estruturas de correção estão apresentados na \autoref{tab:cov}. 
Os resultados indicam que não diferenças significativas entre as estruturas CS, Exp e Gaus. Na realidade, com base
nos valores de AIC e BIC verifica-se que a estrutura CS é a mais adequada. Este fato é verificado também por meio
do teste da razão de verossimilhança, o qual verifica se o parâmetro adicional das estruturas Exp e Gaus não são 
significativos para o modelo.

\begin{table}[H]
	\centering
	\caption{Comparação entre as estruturas de correlação.}
	\onehalfspacing
	\label{tab:cov}
\begin{tabular}{cccc}\toprule
	Estrutura & G.L. & AIC & BIC \\ \midrule
	CS        & 5    & 2097.613 & 2117.545\\
	Exp       & 6    & 2099.613 & 2123.532\\
	Gaus      &  6   & 2099.613  &2123.532 \\ \midrule
	\multicolumn{2}{c}{Comparação} & TRV & valor-\emph{p} \\ \midrule
	\multicolumn{2}{c}{CS vs Exp} & 3.231$\cdot10^{-7}$ & 0.9995\\ 
	\multicolumn{2}{c}{CS vs Gaus} & 3.231$\cdot10^{-7}$ & 0.9995\\ \bottomrule
\end{tabular}
\end{table}

Tendo em vista os resultados discutidos o modelo especificado em \eqref{eq:mod2} foi escolhido. As estimativas dos
parâmetros e seus respectivos erros padrão estão apresentados na \autoref{tab:res}. No que se refere ao parâmetros 
referente a covariável tempo nota-se que sua estimativa foi de 0.0364 com intervalo de confiança variando entre 
0.0058 e 0.0670, isto indica que a cada 7 dias após a avaliação espera-se um aumento entre 0.0058 a 0.0670 no diâmetro
do fruto.

\newpage
Em relação aos parâmetros $\sigma^2$ e $\sigma^2_\delta$ referente na \autoref{tab:res} estão apresentados
suas estimativas e os intervalos de confiança baseados na função de verossimilhança perfilada. Dada as estimativas
de $\sigma^2$ e $\sigma^2_\delta$ podemos verificar que o coeficiente de correlação entre duas observações da 
mesma arvore é 12.87\%.

\begin{table}[H]
	\centering
	\caption{Resumo do ajuste considerando o modelo \eqref{eq:mod2}.}
	\label{tab:res}
	\onehalfspacing
	\begin{tabular}{lcccccc}\toprule
		Parâmetro  & Estimativa & EP     & LI     & LS         & $t$ & P$(T > t)$  \\\midrule
		$\beta_0$  & 69.6161   & 0.2987  &69.0308 & 70.2013 & 233.0742  &  <0.0001 \\
		$\beta_1$  & 0.0364    & 0.0156  & 0.0058 & 0.0670   & 2.3322 & 0.0203 \\
		$\sigma^2$ & 9.5441    & ---     &  8.1807 & 11.1549 & --- & --- \\
		$\sigma^2_\delta$ & 1.4106 &---   &  0.4733& 2.6686& --- & --- \\\bottomrule
	\end{tabular}
\end{table}
               
Com o intuito de realizar uma critica sobre o modelo ajustado, isto uma análise de resíduo e diagnóstico, 
algumas ferramentas gráficas foram utilizadas para detectar possíveis afastamentos das suposições feitas pelo modelo
definido em \eqref{eq:mod2}. 

Na \autoref{fig:hnp} temos o conhecido \emph{half-normal plot} para os resíduos marginais, note que nenhuma observação
esteve fora do envelope simulado o que por diversos autores caracteriza um ajuste satisfatório do modelo de regressão estabelecido aos dados em estudo.
\begin{figure}[H]
	\centering
	\includegraphics[scale = 0.55]{hnp.pdf}
	\caption{Gráfico meio normal de probabilidades com envelope simulado para os resíduos marginais.}
	\label{fig:hnp}
\end{figure}

\newpage
A \autoref{fig:pred} apresenta o gráfico dos resíduos marginais versus valores preditos. Verifica-se um 
comportamento aleatório em torno de zero, o que caracteriza um resíduo bem comportado.

\begin{figure}[H]
	\centering
	\includegraphics[scale = 0.55]{pred.pdf}
	\caption{Gráfico dos resíduos marginais versus valores preditos.}
	\label{fig:pred}
\end{figure}

No modelo \eqref{eq:mod2} assumimos que o efeito da arvore segue uma distribuição normal. Assim, devemos investigar
os BLUPs do efeito aleatório devido a arvore. Nesse sentido apresentamos a \autoref{fig:qq-ranef}, a qual compara
os resíduos de efeito aleatório da arvore com o percentil teórico da distribuição Normal. Em geral, pode-se notar
que não fortes afastamentos da distribuição dos efeitos aleatórios.

\begin{figure}[H]
	\centering
	\includegraphics[scale = 0.55]{qq-ranef.pdf}
	\caption{QQ-Plot para os resíduos de efeito aleatório da arvore.}
	\label{fig:qq-ranef}
\end{figure}

Finalizando a análise de apresentamos a Figura \ref{fig:cook} na qual apresenta o gráfico da distância de Cook
versus os índices das observações. Nota-se que somente uma observação se destaca das demais com uma distância de
Cook próxima de 0.015. Todavia, conforme critério proposto por \citeonline{Cook1982} não temos problemas de observações
influentes, uma vez que nenhuma observação apresentou distância maior que 1.
 
\begin{figure}[H]
	\centering
	\includegraphics[scale = 0.55]{cook.pdf}
	\caption{Gráfico da distância de Cook versus as observações.}
	\label{fig:cook}
\end{figure}

\newpage
\section{Conclusão}
Diante dos resultados apresentados e discutidos neste relatório conclui-se que: 
\begin{enumerate}[(i)]
	\item não há efeito de tratamento sob o diâmetro do fruto de laranja, fato este verificado na análise descritiva e pelo modelo ajustado; 
	\item após extensa comparação entre as estruturas de correção a \emph{Compound Symmetry} se mostrou a mais adequada
	conforme o teste da razão de verossimilhanças e os critérios AIC e BIC;
	\item o modelo escolhido considerou apenas o tempo (avaliação) como significativo, sendo constatado que a cada 7 dias após a avaliação espera-se um aumento entre 0.0058 a 0.0670 no diâmetro do fruto;
	\item o modelo escolhido apresentou ajuste satisfatório, conforme foi verificado pela análise de resíduos (ver Figuras \ref{fig:hnp}, \ref{fig:pred}, \ref{fig:qq-ranef} e \ref{fig:cook}) . 
\end{enumerate}

%(i) não há efeito de tratamento 
%sob o diâmetro do fruto de laranja, fato este verificado na análise descritiva e pelo modelo ajustado; 
%(ii) após extensa comparação entre as estruturas de correção a \emph{Compound Symmetry} se mostrou a mais adequada
%conforme o teste da razão de verossimilhanças e os critérios AIC e BIC; (iii) o modelo escolhido considerou apenas 
%o tempo, isto é, os dias após a avaliação como covariável, sendo constatado que a cada 7 dias após a avaliação 
%espera-se um aumento entre 0.0058 a 0.0670 no diâmetro do fruto; (iv) o modelo escolhido apresentou ajuste 
%satisfatório, conforme foi verificado pelos gráficos apresentados. 

Finalmente, deve-se destacar que os resultados apresentados na modelagem já haviam sido observado na análise 
exploratório dos dados (ver Figuras \ref{fig:bp1}, \ref{fig:bp2} e \ref{fig:bp3}). Portanto, este trabalhou
mostrou a importância da análise descritiva preliminar a metodologia estatística mais sofisticada.

%, neste o teoria
%dos modelos mistos.



\newpage
\bibliography{refMM}

\newpage
\section*{Apêndice}
Neste apêndice disponibilizo os códigos em \textsf{R} utilizados para a análise dos dados.

\begin{lstlisting}[breaklines = true, frame = tblr, language = SAS, numbers=left, numberstyle=\tiny]
# Definicoes gerais -------------------------------------------------------
rm(list = ls(all.names = TRUE))
bib <- c('lme4', 'lmerTest', 'lsmeans', 'hnp', 'dplyr', 'ggplot2', 'RLRsim', 'nlme', 'xtable', 'influence.ME')
sapply(bib, require, character.only = T)
dados <- read.table(file = 'planta-final.txt', sep = ',', header = T)
dados$tempo_f <- factor(dados$tempo_f)
dados$arvore  <- factor(dados$arvore)
head(dados)
str(dados)

# Descritiva --------------------------------------------------------------

boxplot(diametro ~ trat, data = dados,  xlab = '', ylab = '', cex = 0.6, col = 'gray')
mtext("Dose", side = 1, line = 2.0, cex = 1.8)
mtext("Diametro (mm)", side = 2, line = 2, cex = 1.8)

boxplot(diametro ~ tempo_f, data = dados,  xlab = '', ylab = '', cex = 0.6, col = 'gray')
points(x = unique(dados$tempo_f), y = tapply(dados$diametro, dados$tempo_f, mean), pch = 16, col = 'red', cex = 0.8)
mtext("Dias apos a avaliacao", side = 1, line = 2.0, cex = 1.8)
mtext("Diametro (mm)", side = 2, line = 2, cex = 1.8)

dados %>% ggplot(aes(x = tempo_f, y = diametro, group = interaction(tempo_f, trat))) +
geom_boxplot(aes(fill = factor(trat)), color = 'black') +
stat_summary(aes(group = 1), fun.y = mean, geom="line", color = 'black') +
stat_summary(aes(group = 1), fun.y = mean, geom="point", color = 'gold') +
labs(y = 'Diametro (mm)', x = 'Dias apos a avaliacao', fill = 'Dose: ') +
theme_bw() +
theme(text = element_text(size=20), panel.grid.minor = element_blank(), legend.position="top",
panel.grid.major = element_line(size = 0.4, linetype = 'dotted', colour = 'gray'))

# Ajuste dos modelos ------------------------------------------------------

mod1 <- lme(fixed = diametro ~ trat + tempo_n + trat*tempo_n, data = dados, random = ~ 1 | arvore)
mod2 <- lme(fixed = diametro ~ trat + tempo_n, data = dados, random = ~ 1 | arvore)
mod3 <- lme(fixed = diametro ~ tempo_n, data = dados, random = ~ 1 | arvore)
print(xtable(anova(mod2), digits = 4))

# Comparacoes das estrutura de correlacao ---------------------------------

m.CS <- lme(fixed = diametro ~ trat + tempo_n + trat * tempo_n, data = dados, random = ~ 1 | arvore, 
correlation = corCompSymm(form = ~tempo_n|arvore))
m.Exp <- lme(fixed = diametro ~ trat + tempo_n + trat * tempo_n, data = dados, random = ~ 1 | arvore,
correlation = corExp(form = ~tempo_n|arvore, nugget = T))
m.Gaus <- lme(fixed = diametro ~ trat + tempo_n + trat * tempo_n, data = dados, random = ~ 1 | arvore,
correlation = corGaus(form = ~tempo_n|arvore, nugget = T))
anova(m.CS, m.Exp)
anova(m.CS, m.Gaus)
anova(m.Exp, m.Gaus)
# Modelo escolhido --------------------------------------------------------

mod3 <- lme(fixed = diametro ~ tempo_n, data = dados, random = ~ 1 | arvore)
summary(mod3)
intervals(mod3)
mod3 <- lmer(diametro ~ tempo_n + (1 | arvore), data = dados)
summary(mod3)
confint(mod3)

# Analise de Residuos -------------------------------------------------------

## Residuos marginais (erro aleatorio)
my.hnp <- hnp(mod3, halfnormal = T, how.many.out = T, paint.out = T, plot = T)
plot(my.hnp, xaxt = 'n', yaxt = 'n', xlab = '', ylab = '', cex = 0.6, ylim = c(0, 10))
mtext("Percentil da N(0, 1)", side = 1, line = 2.0, cex = 1.8)
mtext("Residuos marginais", side = 2, line =2, cex = 1.8)
abline(h = seq(0, 10, l = 5), v=seq(0, 3, l = 5), col = "gray", lty = "dotted")
axis(1, seq(0, 3, l = 5))
axis(2, seq(0, 10, l = 5), FF(seq(0, 10, l = 5), 1))

## Residuos de efeitos aleatorios
r2 <- random.effects(mod3)$arvore
qqnorm(r2[, 1], xaxt = 'n', yaxt = 'n', xlab = '', ylab = '', cex = 0.6, main = ""); qqline(r2[, 1])
mtext("Percentil da N(0, 1)", side = 1, line = 2.0, cex = 1.8)
mtext("Residuos de efeitos aleatorios", side = 2, line =2, cex = 1.8)
abline(h = seq(-1.5, 1.5, l = 5), v=seq(-2, 2, l = 5), col = "gray", lty = "dotted")
axis(2, seq(-1.5, 1.5, l = 5))
axis(1, seq(-2, 2, l = 5), FF(seq(-2, 2, l = 5), 1))

## Ajustado versus residuo
x = fitted(mod3); y = residuals(mod3); Rx = range(x); Ry = range(y)
plot(y ~ x, xlab = '', ylab = '', cex = 0.8, xaxt = 'n', yaxt = 'n')
mtext("Valores ajustados", side = 1, line = 2.0, cex = 1.8)
mtext("Residuos marginais", side = 2, line =2, cex = 1.8)
abline(h = seq(Ry[1], Ry[2], l = 5), v=seq(Rx[1], Rx[2], l = 5), col = "gray", lty = "dotted")
axis(2, seq(Ry[1], Ry[2], l = 5), FF(seq(Ry[1], Ry[2], l = 5), 1))
axis(1, seq(Rx[1], Rx[2], l = 5), FF(seq(Rx[1], Rx[2], l = 5), 1))

## Influencia
lmer3.infl <- influence(mod3, obs=TRUE)
cook <- cooks.distance(lmer3.infl)
x = 1:nrow(dados); y = cooks.distance(lmer3.infl); Rx = range(x); Ry = range(y)
plot(y, xlab = '', ylab = '', cex = 0.8, xaxt = 'n', yaxt = 'n')
mtext("Indice das observacos", side = 1, line = 2.0, cex = 1.8)
mtext("Distancia de Cook", side = 2, line = 2, cex = 1.8)
abline(h = seq(Ry[1], Ry[2], l = 5), v=seq(Rx[1], Rx[2], l = 5), col = "gray", lty = "dotted")
axis(2, seq(Ry[1], Ry[2], l = 5), FF(seq(Ry[1], Ry[2], l = 5), 3))
axis(1, seq(Rx[1], Rx[2], l = 5), FF(seq(Rx[1], Rx[2], l = 5), 0))
graphics.off()

\end{lstlisting}



\end{document}